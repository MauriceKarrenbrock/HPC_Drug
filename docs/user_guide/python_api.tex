% ######################################################################################
% # Copyright (c) 2020-2021 Maurice Karrenbrock                                        #
% #                                                                                    #
% # This is part of the documentation of the HPC_Drug software                         #
% #                                                                                    #
% # This software is open-source and is distributed under the                          #
% # GNU Affero General Public License v3 (agpl v3 license)                             #
% #                                                                                    #
% # A copy of the license must be included with any copy of the program or part of it  #
% ######################################################################################


\section{Python API for Advanced Users}

    In this section I will show the usage of some functions and classes that a common user could find useful for the development of custom pipelines. To get a more detailed knowledge of all the classes and functions of HPC\_Drug checkout the developer guide section.
    
    \subsubsection{Class GetProteinLigandFilesPipeline(Pipeline)}
    
	    It is a subclass of the Pipeline class. Here is an example of instantiation:
	    
	    {\textit from HPC\_Drug import pipelines
	    
		my\_object = pipelines.GetProteinLigandFilesPipeline(\\
		protein = '2gz7',\\
		protein\_filetype = 'cif',\\
		local = 'no',\\
		filepath = None,\\
		ligand = None,\\
		Protein\_model = 0,\\
		Protein\_chain = 'A',\\
		repairing\_method = 'pdbfixer')}
	
		protein is compulsory, and if local = 'yes' (meaning that the protein file is already on your PC and shall not be downloaded from the wwPDB) filepath must be given as a string. For any other option the default is the one written above.
		
		The only public method is execute() and returns a HPC\_Drug.structures.protein.Protein instance containing a repaired PDB file of the protein with its metallic ions, and a list of HPC\_Drug.structures.ligand.Ligand instances for any organic (not trash) ligand found in the structure. The protein PDB will only contain the selected Protein\_model model and Protein\_chain chain.
	
	\subsubsection{Class Structure(object)}
	
	It is the superclass for all the structure classes (like Protein and Ligand), it's not instantiatable (it's contructor raises a NotImplementedError), but implements some common methods that subclasses will inherit.
	
	\paragraph{Method write(file\_name = None, struct\_type = 'biopython')}
	
		Writes the self.structure structure on the file\_name file and will update self.pdb\_file with the new name (if it is omitted will overwrite the existing self.pdb\_file), struct\_type tells the function with which tool self.structure was obtained (biopython, prody) the default is "biopython"
	
	\paragraph{update\_structure(struct\_type = "biopython")}
	
		Updates self.structure parsing self.pdb\_file, struct\_type is the tool you want to use (biopython, prody) default "biopython" 
		
	\subsubsection{Class Protein(HPC\_Drug.structures.structure.Structure)}
	
		It is the Protein class, one of the fundamental classes of the program, it contains any possible information about the protein you are studying, it subclasses the HPC\_Drug.structures.structure.Structure class adding some methods to it, and overwriting the \_\_init\_\_ method:
		\begin{itemize}
			\item protein\_id = the protein wwPDB id (string)
			\item pdb\_file = the PDB or mmCIF file of the protein, default \{protein\_id\}.\{file\_type\}
			\item structure = the Biopython\cite{biopython} or the Prody\cite{prody} structure parsed from the pdb\_file
			\item substitutions\_dict = a dictionary that contains information about the metal binding residues and the cysteines that make a disulf bond
			\item sulf\_bonds = a list of tuples containing the couples of cysteines binding in a disulf bond
			\item seqres = a place where to store the residue sequence if needed
			\item file\_type = can be 'cif' or 'pdb' depending on the protein\_pdb file format (mmCIF or PDB), default 'cif'
			\item model = integer, the model taken in consideration (starts from 0), default 0
			\item chain = string, the PDB chain taken in consideration, default 'A'
			\item gro\_file = the Gromacs\cite{gromacs_ABRAHAM201519} .gro file
			\item top\_file = the Gromacs\cite{gromacs_ABRAHAM201519} .top file
			\item tpg\_file the .tpg file, needed for Orac\cite{orac}
			\item prm\_file the .prm file, needed for Orac\cite{orac}
			\item \_ligands = the organic ligands, it is private but there is a method to get them (see below)	
		\end{itemize}
	
		This is an example instantiation:\\
		\textit{from HPC\_Drug.structures import protein\\
		$\ $\\
		my\_protein = protein.Protein(protein\_id = "2gz7", pdb\_file = "2gz7.cif", file\_type = "cif", model = 0, chain = "A")}
	
		Besides the superclass methods Protein implements the above methods:
		
		\paragraph{Method add\_ligands(Ligand)}
		
			Takes a HPC\_Drug.structures.ligand.Ligand instance and add it to self.\_ligands

		\paragraph{Method clear\_ligands()}

            Clears ALL the ligands stored in self.\_ligands

        \paragraph{Method update\_ligands(ligands)}
        
            Takes an iterable (list, tuple, etc...) containing HPC\_Drug.structures.ligand.Ligand 
            instances and ovewrites self.\_ligand with this new ones (any information about the old ones will be lost)
            
            ligands :: iterable containing the new HPC\_Drug.structures.ligand.Ligand instances

			
		\paragraph{Method get\_ligand\_list()}
		
			Returns a list with the pointers to self.\_ligands (it is not a copy of them so pay attention on what you do)

	\subsubsection{Class Ligand(HPC\_Drug.structures.structure.Structure)}
	
	It is the Ligand class, one of the fundamental classes of the program, it contains any possible information about an organic ligand of the studied protein, it subclasses the HPC\_Drug.structures.structure.Structure class only overwriting the \_\_init\_\_ method:
	\begin{itemize}
		\item resname = the ligand residue name (string) capital letters
		\item pdb\_file = the PDB or mmCIF file of the ligand, default \{resname\}.\{file\_type\}
		\item structure = the Biopython\cite{biopython} or the Prody\cite{prody} structure parsed from the pdb\_file
		\item resnum = integer, the residue number inside the original PDB (or mmCIF) file from which the ligand was or will be extracted ( it is very useful if the ligand has not been extracted yet)
		\item file\_type = can be 'cif' or 'pdb' depending on the protein\_pdb file format (mmCIF or PDB), default 'pdb'
		\item itp\_file = the Gromacs\cite{gromacs_ABRAHAM201519} .itp file
		\item gro\_file = the Gromacs\cite{gromacs_ABRAHAM201519} .gro file
		\item top\_file = the Gromacs\cite{gromacs_ABRAHAM201519} .top file
		\item tpg\_file the .tpg file, needed for Orac\cite{orac}
		\item prm\_file the .prm file, needed for Orac\cite{orac}
	\end{itemize}
	
	This is an example instantiation:\\
	\textit{from HPC\_Drug.structures import ligand\\
		$\ $\\
		my\_ligand = ligand.Ligand(resname = "LIG", pdb\_file = "LIG.pdb", file\_type = "pdb", resnum = 4)}
	
	
    