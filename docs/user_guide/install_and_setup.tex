% ######################################################################################
% # Copyright (c) 2020-2020 Maurice Karrenbrock                                        #
% #                                                                                    #
% # This is part of the documentation of the HPC_Drug software                         #
% #                                                                                    #
% # This software is open-source and is distributed under the                          #
% # GNU Affero General Public License v3 (agpl v3 license)                             #
% #                                                                                    #
% # A copy of the license must be included with any copy of the program or part of it  #
% ######################################################################################


\section{Install and Setup}

	In this section I will give a brief overview on how to install HPC\_Drug and how to set up the python environment.
	
	To install the program you simply have to download it from the GitHub repository: https://github.com/MauriceKarrenbrock/HPC\_Drug and, if you already had a setup environment, you could already run the main.py program (or if you would like to use one of the other scripts in the scripts/ directory remember to copy it in the root one first).
	
	The environment setup is a little bit longer, in fact HPC\_Drug, being a middleware, has a fair amount of dependencies. This numbered list below is one example to get the job done fast and smooth, but you may need to do things differently:
	\begin{enumerate}
		\item download and install miniconda https://docs.conda.io/en/latest/miniconda.html
		
		\item create a conda environment with python 3.6.9 (recommended), scipy\cite{scipy}, numpy\cite{numpy}, and pip: \emph{conda create -n HPC\_Drug python=3.6.9 scipy numpy pip}
		
		\item activate the environment: \emph{conda activate HPC\_Drug}
		
		\item install importlib-resources: \emph{pip install importlib-resources} (it is a back port of importlib for python 3.6)
		
		\item install OpenMM\cite{openmm}: \emph{conda install -c omnia -c conda-forge openmm} (you can find the full installation guide here\\ http://docs.openmm.org/latest/userguide/application.html\#installing-openmm); if you need to use mmCIF files you will have to use their development version: \emph{conda install -c omnia-dev openmm} due to some important bug fixes
		
		\item install pdbfixer\cite{pdbfixer}: download it from github https://github.com/openmm/pdbfixer go in the just installed root directory and type \emph{pip install .}
		
		\item install ProDy\cite{prody}: download it from GitHub https://github.com/prody/ProDy go in the just installed root directory and type \emph{pip install .}
		
		\item install Biopython\cite{biopython}: download it from GitHub https://github.com/biopython/biopython go in the just installed root directory and type \emph{pip install .}
		
		\item install plumed\cite{plumed}: if you don't need some of the advanced functionalities of plumed (we won't need them) simpli type \emph{conda install -c conda-forge plumed}, otherwise you can find all the needed information on the plumed website https://www.plumed.org
		
		\item install primadorac\cite{primadorac} and Orac\cite{orac} (they are distributed together): download Orac from it's website http://www.chim.unifi.it/orac and follow the installation guide on the documentation. (As it is a quite challenging task below you will find a little help paragraph for this step)
		
		\item install Gromacs\cite{gromacs_ABRAHAM201519}: if you want to use Gromacs instead of Orac as MD program you can install it by following the instructions on the Gromacs website http://www.gromacs.org , in case you have some problems in the compilation process or you need to patch it with plumed (necessary if you want to use a Replica Exchange Method (REM) in older versions, optional in newer versions) I found this blog article very useful https://sajeewasp.com/gromacs-plumed-gpu-linux/ (it is for an old version of Gromacs but still useful).
	\end{enumerate}

	\paragraph{Installing Orac and primadorac}
	
		Installing Orac and primadorac can be a bit user unfriendly so here is a little installation help, before you start download the gfortran compiler:
		\begin{enumerate}
			\item download and unpack the Orac files (containing the primadorac ones too), we will call this directory orac
			
			\item make a directory called \textasciitilde/ORAC/trunk (it MUST be in your home)
			
			\item go to orac/src and tipe \emph{./configure -GNU -FFTW} and then type \emph{make}. A new directory called GNU will have appeared
			
			\item if you want to use OpenMP or MPI redo the previous step with the needed flags ex: \emph{./configure -GNU -FFTW -OMP} and you will see more directories being made.
			
			\item copy the orac/lib directory in \textasciitilde/ORAC/trunk/lib
			
			\item copy any directory you created inside orac/src in \textasciitilde/ORAC/trunk/src
			
			\item check if the program works (use the executable inside \textasciitilde/ORAC/trunk/src/GNU*)
			
			\item download the MOPAC2016.exe executable from the openmopac webpage http://openmopac.net and install it correctly
			
			\item go to orac/tools/primadorac and run \emph{make}
			
			\item go to orac/tools/primadorac/www directory and check if there is an executable called new\_rms, if not use gfortran to crate it by compiling new\_rms.f
			
			\item copy the orac/tools/primadorac directory in \textasciitilde/ORAC/trunk/tools/primadorac
			
			\item check if primadorac works properly\\
			(the right executable is  \textasciitilde/ORAC/trunk/tools/primadorac/primadorac.bash)
			
			\item at his point everything should work file
			
		\end{enumerate}
	
		This shall not be taken as a complete Orac and primadorac installation guide but only as an help.
