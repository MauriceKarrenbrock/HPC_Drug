% ######################################################################################
% # Copyright (c) 2020-2020 Maurice Karrenbrock                                        #
% #                                                                                    #
% # This is part of the documentation of the HPC_Drug software                         #
% #                                                                                    #
% # This software is open-source and is distributed under the                          #
% # GNU Affero General Public License v3 (agpl v3 license)                             #
% #                                                                                    #
% # A copy of the license must be included with any copy of the program or part of it  #
% ######################################################################################


\subsection{HPC\_Drug/PDB/structural\_information/scan\_structure.py}

    This file contains the functions necessary to scan the structure of a PDB file or a headerless mmCIF file

    \subsubsection{Function get\_metal\_binding\_residues\_with\_no\_header(structure, cutoff = 3.0, protein\_chain = 'A', protein\_model = 0, COM\_distance = 10.0, metals = important\_lists.metals)}

        This function gets called by get\_metalbinding\_disulf\_ligands

        This function iterates through the structure many times in order to return the metal binding residues through a substitution dictionary

        {residue\_id : [residue\_name, binding\_atom, binding\_metal]}

        It uses biopython structures

        structure :: a biopython structure of the protein

        cutoff :: double the maximum distance that a residue's center of mass and a metal ion can have to be considered binding default 3.0 angstrom

        protein\_chain :: string default 'A', if == None no chain selection will be done

        protein\_model :: integer default 0, if == None no model and no chain selection will be done

        metals :: a list (or tuple etc) that contains all the resnames (in capital letters) of metals necessary to look for, default HPC\_Drug.important\_lists.metals (Actually the easiest way to personalize metals is to append your custom values to this list)


        this function is slow and error prone and should only be used if there is no mmCIF with a good header

        It should not be necessary to change COM\_distance because it simply is the distance between the center of mass of a residue and the metal that is used to know which atom distances to calculate

    \subsubsection{Function get\_disulf\_bonds\_with\_no\_header(structure, cutoff = 3.0, protein\_chain = 'A', protein\_model = 0)}

        This function gets called by get\_metalbinding\_disulf\_ligands

        This function iterates through the structure many times in order to return the disulf bonds through a substitution dictionary and a list of the binded couples

        {residue\_id : [residue\_name, binding\_atom, binding\_metal]} and [(cys\_id, cys\_id), (cys\_id, ...), ...]

        return substitutions\_dict, sulf\_bonds

        it uses a biopython structure

        structure :: biopython structure of the protein

        cutoff :: double the maximum distance that two CYS S atoms can have to be considered binding default 3.0 angstrom

        protein\_chain :: string default 'A', if == None no chain selection will be done

        protein\_model :: integer default 0, if == None no model and no chain selection will be done

        this function is slow and error prone and should only be used if there is no mmCIF with a good header


    \subsubsection{Function get\_organic\_ligands\_with\_no\_header(structure, protein\_chain = 'A', protein\_model = 0, trash = important\_lists.trash, metals = important\_lists.metals)}

        This function gets called by get\_metalbinding\_disulf\_ligands

        This function iterates through the structure to get the organic ligand

        returning a list of lists containing [[resname, resnumber], [resname, resnumber], ...]

        If there are none returns None

        it uses a biopython structure

        structure :: biopython structure of the protein

        protein\_chain :: string default 'A', if == None no chain selection will be done

        protein\_model :: integer default 0, if == None no model and no chain selection will be done

        trash :: a list (or tuple etc) that contains all the resnames (in capital letters) of trash ligands to avoid listing, default HPC\_Drug.important\_lists.trash (Actually the easiest way to personalize trash is to append your custom values to this list)

        metals :: a list (or tuple etc) that contains all the resnames (in capital letters) of metals necessary to look for, default HPC\_Drug.important\_lists.metals (Actually the easiest way to personalize metals is to append your custom values to this list)

        this function is slow and error prone and should only be used if there is no mmCIF with a good header

    \subsubsection{Function et\_metalbinding\_disulf\_ligands(Protein, trash = important\_lists.trash, metals = important\_lists.metals)}
    
        This is a template that uses the other funcions on this file to return a dictionary with key = resnum and value = (resname, binding atom, metal) or ('CYS', 'SG', 'disulf') depending if the residue number {resnum} binds a metallic ion or is part of a disulf bond and updates Protein.substitutions\_dict with it

        and a list of tuples that contain the couples of CYS that are part of a disulf bond and updates Protein.sulf\_bonds with it

        and a list of tuples with the residue name and residue number of the organic ligands (if there are none None will be returned)

        Protein :: a HPC\_Drug.structures.protein.Protein instance

        trash :: a list (or tuple etc) that contains all the resnames (in capital letters) of trash ligands to avoid listing, default HPC\_Drug.important\_lists.trash (Actually the easiest way to personalize trash is to append your custom values to this list)

        metals :: a list (or tuple etc) that contains all the resnames (in capital letters) of metals necessary to look for, default HPC\_Drug.important\_lists.metals (Actually the easiest way to personalize metals is to append your custom values to this list)

        return Protein, [[lig\_resname, lig\_resnum], ...]