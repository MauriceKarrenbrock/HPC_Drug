% ######################################################################################
% # Copyright (c) 2020-2020 Maurice Karrenbrock                                        #
% #                                                                                    #
% # This is part of the documentation of the HPC_Drug software                         #
% #                                                                                    #
% # This software is open-source and is distributed under the                          #
% # GNU Affero General Public License v3 (agpl v3 license)                             #
% #                                                                                    #
% # A copy of the license must be included with any copy of the program or part of it  #
% ######################################################################################



\subsection{HPC\_Drug/MD/orac/orac\_input.py}

    This file contains the super class of the Orac input templates

    \subsubsection{Class OracInput(object)}

        Super class of any Orac input template object

        \paragraph{Method \_\_init\_\_(self, Protein, solvent\_pdb = None, MD\_program\_path = 'orac')}

            Protein :: HPC\_Drug.structures.protein.Protein instance

            solvent\_pdb :: string, it is the pdb file that contains the coordinates of a solvent molecule 
            it is needed if there has to be added a solvent box around the protein 
            default HPC\_Drug.lib "water.pdb"

            MD\_program\_path :: string, the absolute path to the orac executable 
            dafault will look for an executable called orac in the PATH and the working directory (in this order)

        \paragraph{Method \_write\_box(self)}
            
            private

            Writes the string about the box size
            and rotates the strucure in a smart way

        \paragraph{Method \_write\_sulf\_bond\_string(self)}
        
            private
            
            Writes the part of the template inherent to the disulf bonds


        \paragraph{Method \_get\_protein\_resnumber\_cutoff(self)}
        
            private
            
            Orac starts the residue count from one
            but pdb files often don't
            so I get the id of the resnumber of the first residue
            
            returns the cutoff to apply to start from one

        
        \paragraph{Method \_write\_ligand\_tpg\_path(self)}
            
            private
            
            Writes the tpg file for any given ligand


        \paragraph{Method \_write\_ligand\_prm\_path(self)}
        
            private
            
            Writes the prm file for any given ligand


        \paragraph{Method \_get\_ligand\_name\_from\_tpg(self)}
        
            private
            
            Gets the name of the ligand from the tpg file
            for any given ligand


        \paragraph{Method \_write\_solvent\_grid(self)}
        
            private
            
            Writes the informations about the solvent grid    
            
        
        \paragraph{Method \_write\_EWALD\_PME(self)}
            
            private
            
            Writes the informations about the grid
            in the reciprocal reticle


        \paragraph{Method \_write\_ADD\_STR\_COM(self)}
            
            private
            
            Writes the informations about the bounding between the protein and the ligands


        \paragraph{Method \_write\_LINKED\_CELL(self)}
            
            private
            
            writes the LINKED CELL string


        \paragraph{Method \_write\_template\_on\_file(self)}
        
            private

            Writes the objects template on {filename} file


        \paragraph{Method \_write\_chain\_in\_pdb(self)}
            
            private

            This is a patch because orac removes the chain id from
            pdb files and this confuses some pdb parsers


        \paragraph{Method \_run(self)}
        
            private

        \paragraph{Method execute(self)}
        
            Runs the requested Orac run and returns an updated 
            HPC\_Drug.structures.protein.Protein instance

            (The only pubblic method of this class)